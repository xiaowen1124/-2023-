
\documentclass[a4paper,12pt]{article}  
\usepackage{ctex}
\usepackage{amsmath, amsthm, amsfonts, amssymb,bm}
\usepackage{graphicx}
\usepackage{setspace}
\usepackage{indentfirst}
\usepackage{fontspec}
\usepackage{tabularx}
\usepackage{array}
\usepackage{tabularx}
\usepackage{xcolor}
\usepackage{listings}
\usepackage[margin=1.5cm, left=2cm, right=2cm]{geometry} % 设置页边距


\usepackage[colorlinks=true]{hyperref} % 引入hyperref宏包,并启用颜色链接

\setmainfont{Times New Roman}

% 设置行间距
\setlength{\baselineskip}{20pt}

% 设置段落格式
\setlength{\parindent}{2em} % 首行缩进2个汉字字符
\setlength{\parskip}{0pt} % 段前段后距离为0

% 设置章节标题格式
\usepackage{titlesec}
\titleformat{\section}{\centering\zihao{3}\bfseries}{\thesection}{1em}{}
\titleformat{\subsection}{\zihao{-4}\bfseries}{\thesubsection}{1em}{}


\usepackage[most]{tcolorbox}

\tcbset {
  base/.style={
    arc=0mm,
    bottomtitle=0.5mm,
    boxrule=0mm,
    colbacktitle=black!10!white,
    coltitle=black,
    fonttitle=\bfseries,
    left=2.5mm,
    leftrule=1mm,
    right=3.5mm,
    title={#1},
    toptitle=0.75mm,
  }
}

\definecolor{brandblue}{rgb}{0.34, 0.7, 1}
\newtcolorbox{mainbox}[1]{
  colframe=brandblue,
  base={#1}
}

\newtcolorbox{subbox}[1]{
  colframe=black!30!white,
  base={#1}
}

\begin{document}


\renewcommand{\arraystretch}{1.5}
\begin{center}
    \zihao{-2} \textbf{马克思主义基本原理概论及实践期末试题}\\[0.5cm]
    \zihao{-4}
    \begin{tabularx}{\textwidth}{|>{\centering\arraybackslash}X|>{\centering\arraybackslash}X|>{\centering\arraybackslash}X|>{\centering\arraybackslash}X|}
        \hline
        \textbf{学期} & $2022-2023$ (2) & \textbf{试卷类型} & B \\
        \hline
        \textbf{课程编号} &16402002 & \textbf{考试形式} &闭卷 \\
        \hline
        \textbf{一} &  \textbf{二}& \textbf{三} &  \textbf{总分}\\
        \hline
        题目主要以考研政治题为主,复习考研题可取得好成绩(题目及答案源自网络) & 记得去年是6月19日考的马原(结合了老师画的重点看了一下考研真题)& 题目基本是回忆之后找到的原题,可能有些许出入,答案仅供参考 & \\
        \hline
    \end{tabularx}
\end{center}




\section{单项选择题(每题2分共计20分)}



1. 1844年8月底, 马克思和恩格斯在巴黎会面。这次会面使他们发现彼此的基本观点完全一致, 于是开始了在科学理论研究和革命活动中的长期合作。马克思和恩格斯首次系统阐述历史唯物主义基本观点的著作是(A)


%$ A $ 选项, 1844 年, 马克思和恩格斯合写了《德意志意识形态》, 首次系统阐述了历史唯物主义的基本观点, 实现了历史观上的伟大变革。 $ B $ 选项, 《神圣家族》是马克思和恩格斯第一次合写的批判青年黑格尔派主观唯心主义和论述历史唯物主义的著作。 $ C $ 选项, 1847 年, 马克思撰写并发表了《哲学的贫困》, 以批判蒲鲁东在 1846 年发表的《贫困的哲学》。该著作是马克思主义学说最早发表的文本。 $ D $ 选项, 《共产党宣言》发表于 1848 年, 标志着马克思主义的公开问世。因此, 本题的正确答案是 $ A $ 选项。


\noindent\begin{tabular}{l}
  % after \\: \hline or \cline{col1-col2} \cline{col3-col4} ...
$ A $. 《德意志意识形态》 \\
$ B $. 《神圣家族》 \\
$ C $.《哲学的贫困》 \\
$ D $. 《共产党宣言》 \\
\end{tabular}

2. 时间是万事万物存在的刻度,1秒钟,电影放映 24 帧画面,猎豹在草原上飞奔 28 米,蜂鸟振动翅膀 55 次;1分钟,登山队员攀登珠峰顶峰58.3厘米, “复兴号”前进 5833 米。时间创造无限可能。有人努力奔跑,用全力以赴的冲刺突破极限;有人砥砺前行,以日复一日的坚守辛勤耕耘。原本匀速流动的时间,正是在生生不息的奋斗中,在昂扬奋发的进取中,确定意义、体现价值、进而定义生命的精彩、定格历史的脉动。人们在奋斗中“定义”时间, 下面关于时间的表述中错误的是(C)。\textcolor{red}{考察时间观}

\noindent\begin{tabular}{l}
$ A $.测量事物运动的客观尺度\\
$ B $.物质运动的存在形式\\
$ C $.事物运动的主观联想\\
$ D $.与物质运动不可分割的\\
\end{tabular}\\

%物质运动总是在一定的时间和空间中进行的,没有离开物质运动的“纯粹”时间和空间,也没有离开时间和空间的物质运动。物质运动与时间和空间的不可分割, 证明了时间和空间的客观性。具体物质形态的时空是有限的, 而整个物质世界的时空是无限的。故 $ C $ 错误。 $ A D $ 正确。时间和空间是物质运动的存在形式, 时间是指物质运动的持续性、顺序性, 特点是一维性。故 $ B $ 正确。


3. 在深入推动黄河流域生态保护和高质是发展座谈会上, 近平总书记谈及水资源和发展的关系时,以传统名吃“羊肉泡馍”作形象比喻, 强调要全方位贯彻 “四水四定”(以水定城、以水定地、以水定人、以水定产)原则,精打细算用好水资源,“有多少汤泡多少馍”,让水资源用在最该用的地方。下列不属于 “有多少汤泡多少馍”蕴含的哲学道理是(B)。\textcolor{red}{考察主观能动性和客观规律的统一性}

\noindent\begin{tabular}{l}
$ A $. 一切从实际出发, 实事求是\\
$ B $. 创造条件, 充分发挥意识能动性\\
$ C $. 因地制宜,因时制宜\\
$ D $. 尊重规律,把握适度原则\\
\end{tabular}\\




4. 我国数学家华罗庚在一次报告中以 “一支粉笔多长为好”为例来讲解他所倡导的选法,对此,他解释道:“每支粉笔都要丢掉一段一定长的粉笔头, 但就这一点来说愈长愈好。但太长了, 使用起来很不方便,而且容易折断。每断一次,必然多浪费一个粉笔头,反而不合适。因而就出现了粉笔多长最合适的问题一一这就是一个优选问题。”所谓优选问题,从辩证法的角度看,就是要(C)。\textcolor{red}{考察唯物辩证法}


\noindent\begin{tabular}{l}
$ A $. 注重量的积累\\
$ B $. 保持事物质的稳定性\\
$ C $. 坚持适度原则\\
$ D $. 全面考虑事物属性的多样性\\
\end{tabular}\\


5.2018 年 3 月 14 日, 著名物理学家霍金逝世。霍金一生中有包括奇点定理和霍金辐射在内的一系列重要发现。虽然他关于霍金辐射的理论已经被全球物理学家普遍接受, 也让物理学家更加重视研究量子力学和爱因斯坦相对论的融合,但霍金辐射还没有被观测到,也就是说霍金辐射理论还没有被证实, 这也是霍金没有能够获得诺贝尔奖的原因之一。这说明(A)。\textcolor{red}{考察方法论}

\noindent\begin{tabular}{l}
$ A $. 认识的真理性必须接受实践的检验\\
$ B $. 检验真理的实践活动是逻辑证明的产物\\
$ C $. 追求真理是认识的最终目的\\
$ D $. 真理是在一定条件下的认识\\
\end{tabular}\\


6. 习近平强调, “历史是最好的教科书”, “历史的经验值得注意,  历史的教训更应引以为戒”,“中国革命历史是最好的营养剂”。 人们能够从历史中汲取经验教训,是因为( B)。\textcolor{red}{考察社会基本矛盾及其运动规律}

\noindent\begin{tabular}{l}
$ A $. 历史规律和自然规律存在着惊人的相似性\\
$ B $. 人类历史发展存在着不以人的意志为转移的规律\\
$ C $. 历史总是在循环往复中不断向前发展\\
$ D $. 人类已经完全掌握了历史发展的内在规律\\
\end{tabular}\\

7. 坚持以人民为中心, 就必须坚持人民主体地位, 坚持立党为公、  执政为民,践行全心全意为人民服务的根本宗旨,把党的群众路线 贯彻到治国理政全部活动之中,把人民对美好生活的向往作为奋斗 目标。“坚持以人民为中心”的理论基础是唯物史观关于 (D) 。\textcolor{red}{考察唯物史观}


\noindent\begin{tabular}{l}
$ A $.  总体的人在总体的历史过程中的主体地位的原理\\
$ B $. 人的本质是一切社会关系的总和的原理\\
$ C $. 人民群众的活动受到社会历史条件制约的原理\\
$ D $. 人民群众是历史的创造者的原理\\
\end{tabular}\\


8. 社会形态是关于社会运动的具体形式、发展阶段和不同质态的范  畴。是同生产力发展一定阶段相适应的经济基础同上层建筑的统一体。人类社会历史划分为原始社会、奴隶社会、封建社会、资本主义社会和共产主义社会 (社会主义社会是其第一阶段) 五种社会形态, 其依据是 ( C)。\textcolor{red}{考察社会形态}


\noindent\begin{tabular}{l}
$ A $.  生产工具的质量和数量\\
$ B $. 统治集团的阶级和政治属性\\
$ C $. 经济基础特别是生产关系的性质\\
$ D $. 人们的社会交往和分工的范围和水平\\
\end{tabular}




9. 马克思指出: “我们从小麦的滋味中尝不出种植小麦的人是俄国的衣奴,法国的小农,还是英国的资本家。”这说明( D)。\textcolor{red}{考察商品的使用价值}

\noindent\begin{tabular}{l}
$ A $.  使用价值是历史范畴\\
$ B $. 物品的使用价值都是劳动者生产出来的\\
$ C $. 同一物品的使用价值随着生产关系的变化而变化\\
$ D $. 同一物品的使用价值并不反映生产关系的性质\\
\end{tabular}



10. 在以私有制经济为基础的商品经济中, 商品生产者的私人劳动生  产的产品是否与社会的需求相适应, 作为具体劳动的有用性质能否 为社会所承认,商品的使用价值和价值之间的矛盾是否能得到解决.  决定着商品生产者的命运。以私有制为基础的商品经济的基本矛盾是(B)。

\noindent\begin{tabular}{l}
$ A $.  使用价值和价值的矛盾\\
$ B $. 私人劳动和社会劳动的矛盾\\
$ C $. 具体劳动和抽象劳动的矛盾\\
$ D $. 脑力劳动和体力劳动的矛盾\\
\end{tabular}

\newpage
\section{材料题(每题  25 分, 共 50 分)字数不少于 400 字}
%结合材料回答问题, 要求每题答题字数不少于 400 字。(每题  25 分, 共 50 分)

\subsection{问题一}
推进马克思主义中国化时代化是一个追求真理、揭示真理、笃行真理的过程。十八 大以来, 国内外形势新变化和实践新要求, 迫切需要我们从理论和实践的结合上深入回 答关系党和国家事业发展、党治国理政的一系列重大时代课题。我们党勇于进行理论探 索和创新, 以全新的视野深化对共产党执政规律、社会主义建设规律、人类社会发展规律 的认识, 取得重大理论创新成果, 集中体现为新时代中国特色社会主义思想。十九大、十 九届六中全会提出的“十个明确”、“十四个坚持”、“十三个方面成就”概括了这一思想的 主要内容, 必须长期坚持并不断丰富发展。

中国共产党人深刻认识到, 只有把马克思主义基本原理同中国具体实际相结合、同 中华优秀传统文化相结合, 坚持运用辩证唯物主义和历史唯物主义, 才能正确回答时代 和实践提出的重大问题, 才能始终保持马克思主义的蓬勃生机和旺盛活力。

坚持和发展马克思主义, 必须同中国具体实际相结合。我们坚持以马克思主义为指 导, 是要运用其科学的世界观和方法论解决中国的问题, 而不是要背诵和重复其具体结 论和词句, 更不能把马克思主义当成一成不变的教条。我们必须坚持解放思想、实事求 是、与时俱进、求真务实, 一切从实际出发, 着眼解决新时代改革开放和社会主义现代化 建设的实际问题, 不断回答中国之问、世界之问、人民之问、时代之问, 作出符合中国实际 和时代要求的正确回答, 得出符合客观规律的科学认识, 形成与时俱进的理论成果, 更好 指导中国实践。

坚持和发展马克思主义, 必须同中华优秀传统文化相结合。只有植根本国、本民族 历史文化沃土, 马克思主义真理之树才能根深叶茂。中华优秀传统文化源远流长、博大 精深, 是中华文明的智慧结晶, 其中蕴含的天下为公、民为邦本、为政以德、革故鼎新、任 人唯贤、天人合一、自强不息、厚德载物、讲信修睦、亲仁善邻等, 是中国人民在长期生产 生活中积累的宇宙观、天下观、社会观、道德观的重要体现, 同科学社会主义价值观主张 具有高度契合性。我们必须坚定历史自信、文化自信, 坚持古为今用、推陈出新, 把马克 思主义思想精髓同中华优秀传统文化精华贯通起来、同人民群众日用而不觉的共同价值 观念融通起来, 不断赋予科学理论鲜明的中国特色, 不断夯实马克思主义中国化时代化 的历史基础和群众基础, 让马克思主义在中国牢牢扎根。

摘自 习近平《高举中国特色社会主义伟大旗帜 为全面建设社会主义现代化国家而团结奋斗——在中国共产党第二十次全国代表大会上的报告》

(1) 从实践与认识的关系以及认识的过程角度, 分析为什么说 “推进马克思主义中国化 时代化是一个追求真理、揭示真理、笃行真理的过程”。

(2) 结合材料,运用马克思主义矛盾学说, 分析为什么坚持和发展马克思主义 “必须同中国具体实 际相结合”。\textcolor{red}{(注:考试只考了第二问)}

\begin{mainbox}{参考答案}
(1) 辩证唯物主义认为, 在实践和认识之间, 实践是认识的基础, 实践在认识活动中起着 决定性的作用。第一, 实践是认识的来源。认识的内容是在实践活动的基础上产生和发展 的。第二, 实践是认识发展的动力。实践的需要是推动认识在深度和广度上不断发展之根 本。第三, 实践是认识的目的。认识活动的目的并不在于认识活动本身, 而在于更好地改造 客体, 更有效地指导实践。第四, 实践是检验认识真理性的唯一标准。

人们认识一定事物的过程, 是一个从实践到认识, 再从认识到实践的过程。实践、认识、 再实践、再认识的过程, 也就是通过实践而发现真理, 又通过实践而证实和发展真理的过程。 认识的目的和任务, 就在于获得真理, 并在真理的指导下去改造世界。习近平新时代中国特 色社会主义思想, 是从新时代中国特色社会主义全部实践中产生的理论结晶, 又在指导中国 特色社会主义建设中彰显强大的真理力量和实践力量。

(2) 马克思主义认为, 矛盾具有普遍性, 即矛盾无处不在、无时不有。矛盾又具有特殊 性, 即各个具体事物的矛盾及每一个矛盾的各个方面在发展的不同阶段上各有其特点。矛 盾的特殊性决定了事物的不同性质。只有具体分析矛盾的特殊性, 才能认清事物的本质和 发展规律, 并采取正确的方法和措施去解决矛盾, 推动事物的发展。矛盾的普遍性和特殊性 是辩证统一的关系。矛盾的普遍性即矛盾的共性, 矛盾的特殊性即矛盾的个性。矛盾的共 性是无条件的、绝对的, 矛盾的个性是有条件的、相对的。任何现实存在的事物都是共性与 个性的有机统一, 共性寓于个性之中, 没有离开个性的共性, 也没有离开共性的个性。矛盾

的普遍性和特殊性辩证关系的原理是马克思主义基本原理同各国具体实际相结合的哲学基 础。中国共产党坚持把马克思主义基本原理同中国具体实际相结合、同中华优秀传统文化 相结合, 在推进马克思主义中国化的进程中不断取得革命、建设、改革的新的胜利。
\end{mainbox}

\subsection{问题二}
党的二十大报告指出:“全面建设社会主义现代化国家最艰巨最繁重的任务仍然在农”建设农业强国是一项长期而艰巨的历史任务,要确保中国人的饭碗牢牢端在自己手中, 就要深入实施 “藏粮于地, 藏粮于技” 战略, 加强高标准农田建设和中低产田改造,  综合利用盐碱地。

盐碱地改良是世界性的难题和复杂的系统工程, 必须坚持系统观念。土壤结构、盐分、 微生物群落、作物品种、水利等众多因素交织在一起, 互相影响, 只有将它们全部调整到 最佳状态, 作物才能健康生长, 盐碱地才能变成生态良田。

盐碱土壤颗粒细, 无正常土壤的团粒结构, 板结、干硬、不透水、不透气, 盐碱难以 随水洗掉, 于是科学家从改变土壤团粒结构出发, 发明一种生物基的 “粘结剂”, 将细小 的盐碱土壤颗粒粘结成大颗粒, 人造一种"团粒结构", 土壤通透性提高了, 盐碱就能被快 速地淋洗掉, 促进土壤团聚体的形成。科学家发现, 如果工程设施和土壤改良不配套, 缺 少整体系统化的治理方案, 不仅作物不具备生长条件, 还会不断重复 “脱盐、返盐” 的问 题, 改良周期长。为了解决这一问题, 科学家把实验室建在田间地头, 优化各种技术参数,  在各盐碱区域建立起不同治理模式。

在天津、江苏、山东, 针对土壤易受到海潮侵蚀, 改良过的土壤容易重新返盐, 科学 家就筑堤建闸, 控制地下水位, 防止返盐, 改良土壤; 在山西、内蒙古, 针对矿化水灌溉、 渠道渗慢抬高地下水位, 造成盐化土壤带, 导致盐碱与干旱并存, 科学家就建立灌排系统,  控制地下水位, 将盐分导出, 重塑上壤结构、快速脱除土壤耕作层盐分, 保持土壤水分。 科学家还针对盐碱土壤的组成和肥力, 开发出专用功能性材料, 抗盐碱种子处理剂和抗逆 材料, 解决盐碱土微生物群落结构单一问题, 开展到盐品种筛选等, 最终创建了以"重塑土 壤, 高效脱盐, 疏堵结合, 垦造良田"为原则的生态修复盐碱地系统工程技术体系。

——摘编自《人民日报》(2022 年 7 月 4 日, 10 月 26 日)

(1)从实践与认识的辩证运动角度, 说明科学家将 “盐碱荒地” 改造成”生态良田”所体现的 认识论原理。

(2)为什么盐碱地改良 “必须坚持系统观念”?  \textcolor{red}{(注:考试只考了第一问)}

\begin{mainbox}{参考答案}
分析:    (1)考查认识运动的基本规律。从实践到认识、从认识到实践, 实践、认识、再实践、 再认识, 认识运动不断反复和无限发展, 这是人类认识运动的基本规律。(2)考查系统思维能力。运用系统观念是提高系统思维能力的要求。系统思维能力就 是从事物相互联系的各个方面及其结构和功能进行系统思考的能力, 就是全面系统地分析 和处理问题的能力。

 (1)科学家将“盐碱荒地”改造成“生态良田”既体现了体现了实践与认识的辩证关系原理,又体现了认识运动基本规律原理。首先,实践是认识的基础,对认识起着决定作用。实践是认识的来源,是认识发展的动力,是认识的目的,是检验认识真理性的唯一标准。在盐碱荒地的治理中,科学家多次从实践中遇到的新问题出发,研发出了一系列抗盐碱技术。其次,认识对实践具有能动反作用。正确认识指导实践,会使实践顺利进行,达到预期效果。正是因为科学家运用了正确的系统观念指导盐碱荒地的改造,才能取得一系列治理成果。再次,实践创新是理论创新的动力源泉,理论创新是实践创新的行动指南。科研团队正是结合各地盐碱治理的创新经验,才形成了“重塑土壤,高效脱盐,疏堵结合,垦造良田”为原则的生态修复盐碱地系统工程技术体系。最后,实践与认识的辩证运动,是一个由感性认识到理性认识,又由理性认识到实践的飞跃,是实践、认识、再实践、再认识,循环往复以至无穷的辩证发展过程。在实现最初的技术突破后科学家发现,如果工程设施和土壤改良不配套,会不断重复“脱盐-返盐”的问题,改良周期长。因此又经历了多次的实践活动,才最终系统解决了“脱盐-返盐”的问题,这正体现了实践与认识的辩证运动。
 
    (2)系统是由许多相互联系、相互作用的要素构成并与周围环境发生关系的具有稳定结构和特定功能的有机整体。系统思维以确认事物的普遍有机联系为前提,进而具体把握事物的系统存在、系统联系与系统规律,遵循以整体性、结构性、层次性、开放性和风险性等为基本内容的思维原则,目的是从整体上把握事物并实现事物结构与功能的优化。系统观念是唯物辩证法普遍联系观点的应有之义,从一定意义上说,普遍联系着的事物本身就是一个系统。在盐碱地治理过程中,土壤结构、盐分、微生物群落、作物品种、水利等众多因素交织在一起,互相影响,只有将它们全部调整到最佳状态,作物才能健康生长,盐碱荒地能变成生态良田。
\end{mainbox}



\section{开放性论述题(字数不少于600字)共30分}

近日,笑果文化公司演员HOUSE(李昊石)在演出中涉嫌侮辱人民军队事件引发热议,有细心网友发现演出中还包括对电影《上甘岭》中人民志愿军形象的恶意污蔑,这是典型的历史虚无主义。所谓历史虚无主义,是指对一个政党、国家和民族以往所取得的历史成绩的全盘否定以及对英模人物的恶意诋毁和公然污蔑。它本质上是披着学术研究和言论自由外衣的反动政治思潮,妄图从根本上否定马克思主义指导地位和中国走向社会主义的历史必然性及否定中国共产党的领导。请运用马克思主义基本原理对此进行批判。


\textbf{参考资料:(下载本文件后可点击下方文字跳转)}
\begin{itemize}
    \item \href{http://theory.people.com.cn/GB/n1/2017/0704/c40531-29380517.html}{鉴往知来 旗帜鲜明反对历史虚无主义}
    \item \href{https://www.dswxyjy.org.cn/n1/2018/0507/c398751-29969190.html}{坚定道路自信理直气壮抵制历史虚无主义}
    \item \href{https://www.dswxyjy.org.cn/n1/2017/0817/c398751-29476941.html}{中国共产党反对"历史虚无主义"的历史考察}
\end{itemize}



\end{document}

